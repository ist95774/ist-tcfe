\section{Theoretical Analysis}
\label{sec:analysis}

In this section, the circuit shown in Figure \ref{fig:rc} is analysed
theoretically, in terms of its response when submited to a voltage in
voltage source $V_a$ and a current in the current source $I_d$, using
Octave.

The circuit consists of four loops, where on the top left loop (loop 1) flows a current $I_a$,
on the top right (loop 2) flows a current $I_b$, on the bottom left (loop 3) a current $I_c$
and on bottom right loop (loop 4) a current $I_d$, all of them assigned to be flowing counterclockwise for
the mesh analysis. The voltage and current sources, $v_a$ and $I_d$, receive countinuous inputs and in order to analyse the circuit
we have to measure the voltage in each node and the current flowing in each loop.
For this purpose we will apply both the Kirchhoff Voltage Law (KVL) and Kirchhoff
Current Law (KCL). \par
Starting the analysis using KVL, we obtain four equations correspondent to each loop:

\begin{equation}
  R_1I_a + R_3(I_a - I_b) + R_4(I_a - I_c) = -v_a ;
  \label{eq:kvl1}
\end{equation}

\begin{equation}
  K_bR_3(-I_a + I_b) - I_b = 0;
  \label{eq:kvl2}
\end{equation}

\begin{equation}
  R_4(-I_a + I_c) + R_6I_c + R_7I_c - K_cI_c = 0;
  \label{eq:kvl3}
\end{equation}

\begin{equation}
  I_d = I_d.
  \label{eq:kvl4}
\end{equation}

Using Octave, we can solve this system of equations easily using matrix operations obtaining the following
solution for the currents:

\begin{table}[h]
  \centering
  \begin{tabular}{|l|r|}
    \hline    
    {\bf Name} & {\bf Value [A or V]} \\ \hline
    Ia & -0.000199\\ \hline 
Ib & -0.000209\\ \hline 
Ic & 0.001001\\ \hline 
Id & 0.001041\\ \hline 
Vb & -0.029341\\ \hline 
Ic & 0.001001\\ \hline 
Ib & -0.000209\\ \hline 
Vc & 8.038843\\ \hline 

  \end{tabular}
  \caption{Results using mesh method}
  \label{tab:tabela1}
\end{table}

As for the analysis using KCL, since we have 8 different nodes we must have 8 different equations
in order to have a solvable system of equations, therefore we obtain the following set of equations:

\begin{equation}
  V_0 = 0;
  \label{eq:kcl1}
\end{equation}

\begin{equation}
  V_1 = V_a;
  \label{eq:kcl2}
\end{equation}

\begin{equation}
  \frac{V_2 - V_1}{R_1} + \frac{V_2 - V_3}{R_2} + \frac{V_2 - V_4}{R_4}= 0;
  \label{eq:kcl3}
\end{equation}

\begin{equation}
  \frac{V_3 - V_2}{R_2} - K_b(V_2 - V_4) = 0;
  \label{eq:kcl4}
\end{equation}

\begin{equation}
  K_b(V_2 - V_4) + \frac{V_5 - V_4}{R_5} = I_d;
  \label{eq:kcl5}
\end{equation}

\begin{equation}
  \frac{V_6}{R_6} + \frac{V_6 - V_7}{R_7} = 0;
  \label{eq:kcl6}
\end{equation}

\begin{equation}
  V_4 - K_c\frac{V_6}{R_6} - V_7 =0;
  \label{eq:kcl7}
\end{equation}

\begin{equation}
  \frac{V_4}{R_4} + \frac{V_4 - V_2}{R_3} + \frac{V_4 - V_5}{R_5} + \frac{V_7 - V_6}{R_7} = -I_d,
  \label{eq:kcl8}
\end{equation}

being \textbf{Equation \ref{eq:kcl1}} referent to node 0,\textbf{Equation \ref{eq:kcl2}} to node 1, \textbf{Equation \ref{eq:kcl3}} to node 2,
\textbf{Equation \ref{eq:kcl4}} to node 3, \textbf{Equation \ref{eq:kcl5}} to node 5, \textbf{Equation \ref{eq:kcl6}} to node 6, \textbf{Equation \ref{eq:kcl7}}
to the linear current dependent voltage source and \textbf{Equation \ref{eq:kcl8}} to the sum of both nodes 4 and 7.

Using Octave, we can solve this system of equations easily using matrix operations obtaining the following
solution for the voltages:

\begin{table}[h]
  \centering
  \begin{tabular}{|l|r|}
    \hline    
    {\bf Name} & {\bf Value [A or V]} \\ \hline
    V0 & 0.000000\\ \hline
V1 & 5.195199\\ \hline
V2 & 4.989875\\ \hline
 V3 & 4.556619\\ \hline
V4 & 5.019215\\ \hline
V5 & 8.853743\\ \hline
V6 & -2.012617\\ \hline
V7 & -3.019628\\ \hline
Ic & 0.001001\\ \hline
Ib & -0.000209\\ \hline

  \end{tabular}
  \caption{Results using nodes method}
  \label{tab:tabela2}
\end{table}

As expected from theory, both methods present the same results as can be seen in tables \ref{tab:tabela1} and \ref{tab:tabela2}.

